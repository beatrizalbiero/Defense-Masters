\begin{frame}{Introdução}
	\framesubtitle{Contexto}
	\begin{itemize}
		\item Inspirada pelo controverso debate sobre a aquisição de verbos irregulares em inglês (\cite{chomsky:1968}, ~\cite{Pinker:1988},
~\cite{Albright2003RulesVA}, ~\cite{kirov:2018}), esta pesquisa tem como objetivo estudar o \textbf{processo de inflexão de verbos irregulares em português} sob a perspectiva do modelo computacional \textbf{Encoder-Decoder}.
		\pause
		\item <1->O debate: modelos de aquisição associativos \textit{vs} por regras.
		\pause
		\item <2->Como verbos irregulares podem ser agrupados de acordo com padrões fonéticos, é possível propor modelos associativos para aprender tais padrões. Alguns exemplos de tais grupos são:
	\end{itemize}
	
	\begin{enumerate}
    \item Bob\textbf{ear}: Bob\textbf{eio}, Bloqu\textbf{ear}: Bloqu\textbf{eio}, Chat\textbf{ear}: Chat\textbf{eio}
    
    \item  Agr\textbf{e}d\textbf{i}r: Agr\textbf{i}do, Cons\textbf{e}gu\textbf{i}r: Cons\textbf{i}go, Ins\textbf{e}r\textbf{i}:Ins\textbf{i}ro
    
    \item Cobrir: C\textbf{u}bro, Dormir: D\textbf{u}rmo, Engolir: Eng\textbf{u}lo
    \end{enumerate}


\end{frame}

\begin{frame}{Introdução: Objetivos}

Construir um modelo Encoder-Decoder para aprender o processo de inflexão de verbos irregulares do português.


\end{frame}
